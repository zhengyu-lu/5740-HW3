% Options for packages loaded elsewhere
\PassOptionsToPackage{unicode}{hyperref}
\PassOptionsToPackage{hyphens}{url}
%
\documentclass[
]{article}
\usepackage{amsmath,amssymb}
\usepackage{lmodern}
\usepackage{iftex}
\ifPDFTeX
  \usepackage[T1]{fontenc}
  \usepackage[utf8]{inputenc}
  \usepackage{textcomp} % provide euro and other symbols
\else % if luatex or xetex
  \usepackage{unicode-math}
  \defaultfontfeatures{Scale=MatchLowercase}
  \defaultfontfeatures[\rmfamily]{Ligatures=TeX,Scale=1}
\fi
% Use upquote if available, for straight quotes in verbatim environments
\IfFileExists{upquote.sty}{\usepackage{upquote}}{}
\IfFileExists{microtype.sty}{% use microtype if available
  \usepackage[]{microtype}
  \UseMicrotypeSet[protrusion]{basicmath} % disable protrusion for tt fonts
}{}
\makeatletter
\@ifundefined{KOMAClassName}{% if non-KOMA class
  \IfFileExists{parskip.sty}{%
    \usepackage{parskip}
  }{% else
    \setlength{\parindent}{0pt}
    \setlength{\parskip}{6pt plus 2pt minus 1pt}}
}{% if KOMA class
  \KOMAoptions{parskip=half}}
\makeatother
\usepackage{xcolor}
\IfFileExists{xurl.sty}{\usepackage{xurl}}{} % add URL line breaks if available
\IfFileExists{bookmark.sty}{\usepackage{bookmark}}{\usepackage{hyperref}}
\hypersetup{
  pdftitle={HW3 Problem 3},
  pdfauthor={Zhengyu Lu},
  hidelinks,
  pdfcreator={LaTeX via pandoc}}
\urlstyle{same} % disable monospaced font for URLs
\usepackage[margin=1in]{geometry}
\usepackage{color}
\usepackage{fancyvrb}
\newcommand{\VerbBar}{|}
\newcommand{\VERB}{\Verb[commandchars=\\\{\}]}
\DefineVerbatimEnvironment{Highlighting}{Verbatim}{commandchars=\\\{\}}
% Add ',fontsize=\small' for more characters per line
\usepackage{framed}
\definecolor{shadecolor}{RGB}{248,248,248}
\newenvironment{Shaded}{\begin{snugshade}}{\end{snugshade}}
\newcommand{\AlertTok}[1]{\textcolor[rgb]{0.94,0.16,0.16}{#1}}
\newcommand{\AnnotationTok}[1]{\textcolor[rgb]{0.56,0.35,0.01}{\textbf{\textit{#1}}}}
\newcommand{\AttributeTok}[1]{\textcolor[rgb]{0.77,0.63,0.00}{#1}}
\newcommand{\BaseNTok}[1]{\textcolor[rgb]{0.00,0.00,0.81}{#1}}
\newcommand{\BuiltInTok}[1]{#1}
\newcommand{\CharTok}[1]{\textcolor[rgb]{0.31,0.60,0.02}{#1}}
\newcommand{\CommentTok}[1]{\textcolor[rgb]{0.56,0.35,0.01}{\textit{#1}}}
\newcommand{\CommentVarTok}[1]{\textcolor[rgb]{0.56,0.35,0.01}{\textbf{\textit{#1}}}}
\newcommand{\ConstantTok}[1]{\textcolor[rgb]{0.00,0.00,0.00}{#1}}
\newcommand{\ControlFlowTok}[1]{\textcolor[rgb]{0.13,0.29,0.53}{\textbf{#1}}}
\newcommand{\DataTypeTok}[1]{\textcolor[rgb]{0.13,0.29,0.53}{#1}}
\newcommand{\DecValTok}[1]{\textcolor[rgb]{0.00,0.00,0.81}{#1}}
\newcommand{\DocumentationTok}[1]{\textcolor[rgb]{0.56,0.35,0.01}{\textbf{\textit{#1}}}}
\newcommand{\ErrorTok}[1]{\textcolor[rgb]{0.64,0.00,0.00}{\textbf{#1}}}
\newcommand{\ExtensionTok}[1]{#1}
\newcommand{\FloatTok}[1]{\textcolor[rgb]{0.00,0.00,0.81}{#1}}
\newcommand{\FunctionTok}[1]{\textcolor[rgb]{0.00,0.00,0.00}{#1}}
\newcommand{\ImportTok}[1]{#1}
\newcommand{\InformationTok}[1]{\textcolor[rgb]{0.56,0.35,0.01}{\textbf{\textit{#1}}}}
\newcommand{\KeywordTok}[1]{\textcolor[rgb]{0.13,0.29,0.53}{\textbf{#1}}}
\newcommand{\NormalTok}[1]{#1}
\newcommand{\OperatorTok}[1]{\textcolor[rgb]{0.81,0.36,0.00}{\textbf{#1}}}
\newcommand{\OtherTok}[1]{\textcolor[rgb]{0.56,0.35,0.01}{#1}}
\newcommand{\PreprocessorTok}[1]{\textcolor[rgb]{0.56,0.35,0.01}{\textit{#1}}}
\newcommand{\RegionMarkerTok}[1]{#1}
\newcommand{\SpecialCharTok}[1]{\textcolor[rgb]{0.00,0.00,0.00}{#1}}
\newcommand{\SpecialStringTok}[1]{\textcolor[rgb]{0.31,0.60,0.02}{#1}}
\newcommand{\StringTok}[1]{\textcolor[rgb]{0.31,0.60,0.02}{#1}}
\newcommand{\VariableTok}[1]{\textcolor[rgb]{0.00,0.00,0.00}{#1}}
\newcommand{\VerbatimStringTok}[1]{\textcolor[rgb]{0.31,0.60,0.02}{#1}}
\newcommand{\WarningTok}[1]{\textcolor[rgb]{0.56,0.35,0.01}{\textbf{\textit{#1}}}}
\usepackage{graphicx}
\makeatletter
\def\maxwidth{\ifdim\Gin@nat@width>\linewidth\linewidth\else\Gin@nat@width\fi}
\def\maxheight{\ifdim\Gin@nat@height>\textheight\textheight\else\Gin@nat@height\fi}
\makeatother
% Scale images if necessary, so that they will not overflow the page
% margins by default, and it is still possible to overwrite the defaults
% using explicit options in \includegraphics[width, height, ...]{}
\setkeys{Gin}{width=\maxwidth,height=\maxheight,keepaspectratio}
% Set default figure placement to htbp
\makeatletter
\def\fps@figure{htbp}
\makeatother
\setlength{\emergencystretch}{3em} % prevent overfull lines
\providecommand{\tightlist}{%
  \setlength{\itemsep}{0pt}\setlength{\parskip}{0pt}}
\setcounter{secnumdepth}{-\maxdimen} % remove section numbering
\ifLuaTeX
  \usepackage{selnolig}  % disable illegal ligatures
\fi

\title{HW3 Problem 3}
\author{Zhengyu Lu}
\date{2022-10-18}

\begin{document}
\maketitle

\hypertarget{hw2-prohlem-3}{%
\subsection{HW2 Prohlem 3}\label{hw2-prohlem-3}}

\hypertarget{a-create-a-binary-variable-mpg01-that-contains-a-1-if-mpg-contains-a-value-above-its-median-and-a-0-if-mpg-contains-a-value-below-its-median.-you-can-compute-the-median-using-the-median-function.-note-you-may-find-it-helpful-to-use-the-data.frame-function-to-create-a-single-data-set-containing-both-mpg01-and-the-other-auto-variables.}{%
\subsubsection{(a) Create a binary variable, mpg01, that contains a 1 if
mpg contains a value above its median, and a 0 if mpg contains a value
below its median. You can compute the median using the median()
function. Note you may find it helpful to use the data.frame() function
to create a single data set containing both mpg01 and the other Auto
variables.}\label{a-create-a-binary-variable-mpg01-that-contains-a-1-if-mpg-contains-a-value-above-its-median-and-a-0-if-mpg-contains-a-value-below-its-median.-you-can-compute-the-median-using-the-median-function.-note-you-may-find-it-helpful-to-use-the-data.frame-function-to-create-a-single-data-set-containing-both-mpg01-and-the-other-auto-variables.}}

\begin{Shaded}
\begin{Highlighting}[]
\FunctionTok{library}\NormalTok{(ISLR)}
\FunctionTok{library}\NormalTok{(MASS)}
\FunctionTok{library}\NormalTok{(e1071)}
\end{Highlighting}
\end{Shaded}

\begin{verbatim}
## Warning: package 'e1071' was built under R version 4.1.2
\end{verbatim}

\begin{Shaded}
\begin{Highlighting}[]
\FunctionTok{attach}\NormalTok{(Auto)}
\FunctionTok{library}\NormalTok{(class)}
\NormalTok{mpg01 }\OtherTok{\textless{}{-}} \FunctionTok{ifelse}\NormalTok{( mpg }\SpecialCharTok{\textgreater{}} \FunctionTok{median}\NormalTok{(mpg), }\AttributeTok{yes =} \DecValTok{1}\NormalTok{, }\AttributeTok{no =} \DecValTok{0}\NormalTok{)}
\NormalTok{Auto }\OtherTok{\textless{}{-}} \FunctionTok{data.frame}\NormalTok{(mpg01,Auto)}
\end{Highlighting}
\end{Shaded}

\hypertarget{b-explore-the-data-graphically-in-order-to-investigate-the-association-between-mpg01-and-the-other-features.-which-of-the-other-features-seem-most-likely-to-be-useful-in-predicting-mpg01-scatterplots-and-boxplots-may-be-useful-tools-to-answer-this-question.-describe-your-findings.}{%
\subsubsection{(b) Explore the data graphically in order to investigate
the association between mpg01 and the other features. Which of the other
features seem most likely to be useful in predicting mpg01? Scatterplots
and boxplots may be useful tools to answer this question. Describe your
findings.}\label{b-explore-the-data-graphically-in-order-to-investigate-the-association-between-mpg01-and-the-other-features.-which-of-the-other-features-seem-most-likely-to-be-useful-in-predicting-mpg01-scatterplots-and-boxplots-may-be-useful-tools-to-answer-this-question.-describe-your-findings.}}

\begin{Shaded}
\begin{Highlighting}[]
\FunctionTok{pairs}\NormalTok{(Auto)}
\end{Highlighting}
\end{Shaded}

\includegraphics{HW3-Problem3-Rmarkdown_files/figure-latex/unnamed-chunk-2-1.pdf}

\begin{Shaded}
\begin{Highlighting}[]
\FunctionTok{boxplot}\NormalTok{(cylinders }\SpecialCharTok{\textasciitilde{}}\NormalTok{ mpg01, }\AttributeTok{data =}\NormalTok{ Auto, }\AttributeTok{main =} \StringTok{"Cylinders vs mpg01"}\NormalTok{)}
\end{Highlighting}
\end{Shaded}

\includegraphics{HW3-Problem3-Rmarkdown_files/figure-latex/unnamed-chunk-2-2.pdf}

\begin{Shaded}
\begin{Highlighting}[]
\FunctionTok{boxplot}\NormalTok{(displacement }\SpecialCharTok{\textasciitilde{}}\NormalTok{ mpg01, }\AttributeTok{data =}\NormalTok{ Auto, }\AttributeTok{main =} \StringTok{"Displacement vs mpg01"}\NormalTok{)}
\end{Highlighting}
\end{Shaded}

\includegraphics{HW3-Problem3-Rmarkdown_files/figure-latex/unnamed-chunk-2-3.pdf}

\begin{Shaded}
\begin{Highlighting}[]
\FunctionTok{boxplot}\NormalTok{(horsepower }\SpecialCharTok{\textasciitilde{}}\NormalTok{ mpg01, }\AttributeTok{data =}\NormalTok{ Auto, }\AttributeTok{main =} \StringTok{"Horsepower vs mpg01"}\NormalTok{)}
\end{Highlighting}
\end{Shaded}

\includegraphics{HW3-Problem3-Rmarkdown_files/figure-latex/unnamed-chunk-2-4.pdf}

\begin{Shaded}
\begin{Highlighting}[]
\FunctionTok{boxplot}\NormalTok{(weight }\SpecialCharTok{\textasciitilde{}}\NormalTok{ mpg01, }\AttributeTok{data =}\NormalTok{ Auto, }\AttributeTok{main =} \StringTok{"Weight vs mpg01"}\NormalTok{)}
\end{Highlighting}
\end{Shaded}

\includegraphics{HW3-Problem3-Rmarkdown_files/figure-latex/unnamed-chunk-2-5.pdf}

\begin{Shaded}
\begin{Highlighting}[]
\FunctionTok{boxplot}\NormalTok{(acceleration }\SpecialCharTok{\textasciitilde{}}\NormalTok{ mpg01, }\AttributeTok{data =}\NormalTok{ Auto, }\AttributeTok{main =} \StringTok{"Acceleration vs mpg01"}\NormalTok{)}
\end{Highlighting}
\end{Shaded}

\includegraphics{HW3-Problem3-Rmarkdown_files/figure-latex/unnamed-chunk-2-6.pdf}

\begin{Shaded}
\begin{Highlighting}[]
\FunctionTok{boxplot}\NormalTok{(year }\SpecialCharTok{\textasciitilde{}}\NormalTok{ mpg01, }\AttributeTok{data =}\NormalTok{ Auto, }\AttributeTok{main =} \StringTok{"Year vs mpg01"}\NormalTok{)}
\end{Highlighting}
\end{Shaded}

\includegraphics{HW3-Problem3-Rmarkdown_files/figure-latex/unnamed-chunk-2-7.pdf}

\begin{Shaded}
\begin{Highlighting}[]
\FunctionTok{boxplot}\NormalTok{(origin }\SpecialCharTok{\textasciitilde{}}\NormalTok{ mpg01, }\AttributeTok{data =}\NormalTok{ Auto, }\AttributeTok{main =} \StringTok{"Origin vs mpg01"}\NormalTok{)}
\end{Highlighting}
\end{Shaded}

\includegraphics{HW3-Problem3-Rmarkdown_files/figure-latex/unnamed-chunk-2-8.pdf}
As the figures show, cylinders, weight, displacement, horsepower seem
most likely to be useful in predicting mpg01. Higer cylinders will have
low mpg,higher weighr will have low mpg, higher displacement will have
low mpg, higher horsepower will have low mpg.

\hypertarget{c-split-the-data-into-a-training-set-and-a-test-set.}{%
\subsubsection{(c) Split the data into a training set and a test
set.}\label{c-split-the-data-into-a-training-set-and-a-test-set.}}

\begin{Shaded}
\begin{Highlighting}[]
\NormalTok{index }\OtherTok{\textless{}{-}} \FunctionTok{sort}\NormalTok{(}\FunctionTok{sample}\NormalTok{(}\FunctionTok{nrow}\NormalTok{(Auto), }\FunctionTok{nrow}\NormalTok{(Auto)}\SpecialCharTok{*}\NormalTok{.}\DecValTok{75}\NormalTok{))}
\NormalTok{train }\OtherTok{\textless{}{-}}\NormalTok{ Auto[index,]}
\NormalTok{test }\OtherTok{\textless{}{-}}\NormalTok{ Auto[}\SpecialCharTok{{-}}\NormalTok{index,]}
\end{Highlighting}
\end{Shaded}

\hypertarget{d-perform-lda-on-the-training-data-in-order-to-predict-mpg01-using-the-variables-that-seemed-most-associated-with-mpg01-in-b.-what-is-the-test-error-of-the-model-obtained}{%
\subsubsection{(d) Perform LDA on the training data in order to predict
mpg01 using the variables that seemed most associated with mpg01 in (b).
What is the test error of the model
obtained?}\label{d-perform-lda-on-the-training-data-in-order-to-predict-mpg01-using-the-variables-that-seemed-most-associated-with-mpg01-in-b.-what-is-the-test-error-of-the-model-obtained}}

\begin{Shaded}
\begin{Highlighting}[]
\NormalTok{lda.model }\OtherTok{\textless{}{-}} \FunctionTok{lda}\NormalTok{(mpg01 }\SpecialCharTok{\textasciitilde{}}\NormalTok{ cylinders}\SpecialCharTok{+}\NormalTok{displacement}\SpecialCharTok{+}\NormalTok{horsepower}\SpecialCharTok{+}\NormalTok{weight, }\AttributeTok{data =}\NormalTok{ train)}
\NormalTok{lda.predict }\OtherTok{\textless{}{-}}  \FunctionTok{predict}\NormalTok{(lda.model, test)}
\FunctionTok{table}\NormalTok{(lda.predict}\SpecialCharTok{$}\NormalTok{class , test}\SpecialCharTok{$}\NormalTok{mpg01)}
\end{Highlighting}
\end{Shaded}

\begin{verbatim}
##    
##      0  1
##   0 44  4
##   1  9 41
\end{verbatim}

\begin{Shaded}
\begin{Highlighting}[]
\FunctionTok{mean}\NormalTok{(lda.predict}\SpecialCharTok{$}\NormalTok{class }\SpecialCharTok{!=}\NormalTok{test}\SpecialCharTok{$}\NormalTok{mpg01 )}
\end{Highlighting}
\end{Shaded}

\begin{verbatim}
## [1] 0.1326531
\end{verbatim}

\hypertarget{e-perform-qda-on-the-training-data-in-order-to-predict-mpg01-using-the-variables-that-seemed-most-associated-with-mpg01-in-b.-what-is-the-test-error-of-the-model-obtained}{%
\subsubsection{(e) Perform QDA on the training data in order to predict
mpg01 using the variables that seemed most associated with mpg01 in (b).
What is the test error of the model
obtained?}\label{e-perform-qda-on-the-training-data-in-order-to-predict-mpg01-using-the-variables-that-seemed-most-associated-with-mpg01-in-b.-what-is-the-test-error-of-the-model-obtained}}

\begin{Shaded}
\begin{Highlighting}[]
\NormalTok{qda.model }\OtherTok{\textless{}{-}} \FunctionTok{qda}\NormalTok{(mpg01 }\SpecialCharTok{\textasciitilde{}}\NormalTok{ cylinders}\SpecialCharTok{+}\NormalTok{displacement}\SpecialCharTok{+}\NormalTok{horsepower}\SpecialCharTok{+}\NormalTok{weight, }\AttributeTok{data =}\NormalTok{ train)}
\NormalTok{qda.predict }\OtherTok{\textless{}{-}}  \FunctionTok{predict}\NormalTok{(qda.model, test)}
\FunctionTok{table}\NormalTok{(qda.predict}\SpecialCharTok{$}\NormalTok{class , test}\SpecialCharTok{$}\NormalTok{mpg01)}
\end{Highlighting}
\end{Shaded}

\begin{verbatim}
##    
##      0  1
##   0 46  7
##   1  7 38
\end{verbatim}

\begin{Shaded}
\begin{Highlighting}[]
\FunctionTok{mean}\NormalTok{(qda.predict}\SpecialCharTok{$}\NormalTok{class }\SpecialCharTok{!=}\NormalTok{test}\SpecialCharTok{$}\NormalTok{mpg01 )}
\end{Highlighting}
\end{Shaded}

\begin{verbatim}
## [1] 0.1428571
\end{verbatim}

\hypertarget{f-perform-logistic-regression-on-the-training-data-in-order-to-predict-mpg01-using-the-variables-that-seemed-most-associated-with-mpg01-in-b.-what-is-the-test-error-of-the-model-obtained}{%
\subsubsection{(f) Perform logistic regression on the training data in
order to predict mpg01 using the variables that seemed most associated
with mpg01 in (b). What is the test error of the model
obtained?}\label{f-perform-logistic-regression-on-the-training-data-in-order-to-predict-mpg01-using-the-variables-that-seemed-most-associated-with-mpg01-in-b.-what-is-the-test-error-of-the-model-obtained}}

\begin{Shaded}
\begin{Highlighting}[]
\NormalTok{glm.model }\OtherTok{\textless{}{-}} \FunctionTok{glm}\NormalTok{(mpg01 }\SpecialCharTok{\textasciitilde{}}\NormalTok{ cylinders}\SpecialCharTok{+}\NormalTok{displacement}\SpecialCharTok{+}\NormalTok{horsepower}\SpecialCharTok{+}\NormalTok{weight, }\AttributeTok{data =}\NormalTok{ train)}
\NormalTok{glm.prob }\OtherTok{\textless{}{-}}  \FunctionTok{predict}\NormalTok{(glm.model, test)}
\NormalTok{glm.predict }\OtherTok{\textless{}{-}} \FunctionTok{rep}\NormalTok{(}\DecValTok{0}\NormalTok{, }\FunctionTok{length}\NormalTok{(glm.prob))}
\NormalTok{glm.predict[glm.prob }\SpecialCharTok{\textgreater{}}\NormalTok{ .}\DecValTok{5}\NormalTok{] }\OtherTok{\textless{}{-}} \DecValTok{1}
\NormalTok{glm.predict}
\end{Highlighting}
\end{Shaded}

\begin{verbatim}
##  [1] 0 0 1 1 1 0 0 0 1 1 1 1 0 0 0 0 1 1 1 1 0 0 0 0 0 0 0 0 0 0 0 1 1 0 1 1 0 0
## [39] 0 0 1 1 1 1 1 0 0 0 0 1 0 1 0 1 0 0 1 1 0 0 0 0 0 1 0 0 0 1 1 1 0 1 0 0 1 1
## [77] 1 1 1 1 1 1 1 1 1 1 1 1 1 1 0 0 1 1 0 0 1 1
\end{verbatim}

\begin{Shaded}
\begin{Highlighting}[]
\FunctionTok{table}\NormalTok{(glm.predict,test}\SpecialCharTok{$}\NormalTok{mpg01)}
\end{Highlighting}
\end{Shaded}

\begin{verbatim}
##            
## glm.predict  0  1
##           0 44  4
##           1  9 41
\end{verbatim}

\begin{Shaded}
\begin{Highlighting}[]
\FunctionTok{mean}\NormalTok{(glm.predict }\SpecialCharTok{!=}\NormalTok{test}\SpecialCharTok{$}\NormalTok{mpg01 )}
\end{Highlighting}
\end{Shaded}

\begin{verbatim}
## [1] 0.1326531
\end{verbatim}

\hypertarget{g-perform-naive-bayes-on-the-training-data-in-order-to-predict-mpg01-using-the-variables-that-seemed-most-associated-with-mpg01-in-b.-what-is-the-test-error-of-the-model-obtained}{%
\subsubsection{(g) Perform naive Bayes on the training data in order to
predict mpg01 using the variables that seemed most associated with mpg01
in (b). What is the test error of the model
obtained?}\label{g-perform-naive-bayes-on-the-training-data-in-order-to-predict-mpg01-using-the-variables-that-seemed-most-associated-with-mpg01-in-b.-what-is-the-test-error-of-the-model-obtained}}

\begin{Shaded}
\begin{Highlighting}[]
\NormalTok{nb.fit }\OtherTok{\textless{}{-}} \FunctionTok{naiveBayes}\NormalTok{(mpg01 }\SpecialCharTok{\textasciitilde{}}\NormalTok{ cylinders}\SpecialCharTok{+}\NormalTok{displacement}\SpecialCharTok{+}\NormalTok{horsepower}\SpecialCharTok{+}\NormalTok{weight, }\AttributeTok{data =}\NormalTok{ train)}
\NormalTok{nb.predict }\OtherTok{\textless{}{-}} \FunctionTok{predict}\NormalTok{(nb.fit, test)}
\FunctionTok{table}\NormalTok{(nb.predict,test}\SpecialCharTok{$}\NormalTok{mpg01)}
\end{Highlighting}
\end{Shaded}

\begin{verbatim}
##           
## nb.predict  0  1
##          0 46  5
##          1  7 40
\end{verbatim}

\begin{Shaded}
\begin{Highlighting}[]
\FunctionTok{mean}\NormalTok{(nb.predict}\SpecialCharTok{!=}\NormalTok{test}\SpecialCharTok{$}\NormalTok{mpg01 )}
\end{Highlighting}
\end{Shaded}

\begin{verbatim}
## [1] 0.122449
\end{verbatim}

\hypertarget{h-perform-knn-on-the-training-data-with-several-values-of-k-in-order-to-predict-mpg01.-use-only-the-variables-that-seemed-most-associated-with-mpg01-in-b.-what-test-errors-do-you-obtain-which-value-of-k-seems-to-perform-the-best-on-this-data-set}{%
\subsubsection{(h) Perform KNN on the training data, with several values
of K, in order to predict mpg01. Use only the variables that seemed most
associated with mpg01 in (b). What test errors do you obtain? Which
value of K seems to perform the best on this data
set?}\label{h-perform-knn-on-the-training-data-with-several-values-of-k-in-order-to-predict-mpg01.-use-only-the-variables-that-seemed-most-associated-with-mpg01-in-b.-what-test-errors-do-you-obtain-which-value-of-k-seems-to-perform-the-best-on-this-data-set}}

\begin{Shaded}
\begin{Highlighting}[]
\NormalTok{train.X }\OtherTok{\textless{}{-}}\NormalTok{train[,}\FunctionTok{c}\NormalTok{(}\DecValTok{3}\NormalTok{,}\DecValTok{4}\NormalTok{,}\DecValTok{5}\NormalTok{,}\DecValTok{6}\NormalTok{)]}
\NormalTok{test.X }\OtherTok{\textless{}{-}}\NormalTok{ test[,}\FunctionTok{c}\NormalTok{(}\DecValTok{3}\NormalTok{,}\DecValTok{4}\NormalTok{,}\DecValTok{5}\NormalTok{,}\DecValTok{6}\NormalTok{)]}
\NormalTok{train.mpg01 }\OtherTok{\textless{}{-}}\NormalTok{train}\SpecialCharTok{$}\NormalTok{mpg01}
\NormalTok{knn.predict }\OtherTok{\textless{}{-}}  \FunctionTok{knn}\NormalTok{(train.X, test.X, train.mpg01, }\AttributeTok{k =} \DecValTok{1}\NormalTok{)}
\FunctionTok{mean}\NormalTok{(knn.predict}\SpecialCharTok{!=}\NormalTok{test}\SpecialCharTok{$}\NormalTok{mpg01 )}
\end{Highlighting}
\end{Shaded}

\begin{verbatim}
## [1] 0.1632653
\end{verbatim}

\begin{Shaded}
\begin{Highlighting}[]
\NormalTok{testError }\OtherTok{\textless{}{-}} \FunctionTok{rep}\NormalTok{(}\ConstantTok{NA}\NormalTok{, }\DecValTok{100}\NormalTok{)}
\ControlFlowTok{for}\NormalTok{ (k }\ControlFlowTok{in} \DecValTok{1}\SpecialCharTok{:}\DecValTok{100}\NormalTok{)\{}
\NormalTok{  knn.pred }\OtherTok{\textless{}{-}} \FunctionTok{knn}\NormalTok{(train.X, test.X, train.mpg01, }\AttributeTok{k =}\NormalTok{ k)}
\NormalTok{  testError[k] }\OtherTok{\textless{}{-}} \FunctionTok{mean}\NormalTok{(knn.pred}\SpecialCharTok{!=}\NormalTok{test}\SpecialCharTok{$}\NormalTok{mpg01)}
\NormalTok{\}}
\NormalTok{testError}
\end{Highlighting}
\end{Shaded}

\begin{verbatim}
##   [1] 0.1632653 0.1122449 0.1122449 0.1632653 0.1530612 0.1224490 0.1428571
##   [8] 0.1632653 0.1734694 0.1428571 0.1734694 0.1734694 0.1836735 0.1632653
##  [15] 0.1632653 0.1734694 0.1734694 0.1734694 0.1734694 0.1734694 0.1734694
##  [22] 0.1734694 0.1734694 0.1734694 0.1734694 0.1734694 0.1734694 0.1632653
##  [29] 0.1632653 0.1734694 0.1632653 0.1734694 0.1632653 0.1632653 0.1632653
##  [36] 0.1632653 0.1632653 0.1632653 0.1632653 0.1632653 0.1632653 0.1530612
##  [43] 0.1530612 0.1530612 0.1632653 0.1632653 0.1530612 0.1530612 0.1632653
##  [50] 0.1632653 0.1530612 0.1530612 0.1530612 0.1530612 0.1530612 0.1428571
##  [57] 0.1530612 0.1530612 0.1530612 0.1428571 0.1428571 0.1530612 0.1530612
##  [64] 0.1428571 0.1428571 0.1428571 0.1428571 0.1530612 0.1530612 0.1632653
##  [71] 0.1530612 0.1632653 0.1530612 0.1530612 0.1530612 0.1530612 0.1632653
##  [78] 0.1632653 0.1530612 0.1428571 0.1428571 0.1326531 0.1428571 0.1326531
##  [85] 0.1326531 0.1326531 0.1326531 0.1326531 0.1326531 0.1326531 0.1530612
##  [92] 0.1530612 0.1530612 0.1530612 0.1530612 0.1530612 0.1632653 0.1632653
##  [99] 0.1530612 0.1530612
\end{verbatim}

\begin{Shaded}
\begin{Highlighting}[]
\FunctionTok{which.min}\NormalTok{(testError)}
\end{Highlighting}
\end{Shaded}

\begin{verbatim}
## [1] 2
\end{verbatim}

\end{document}
